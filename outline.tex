\documentclass{article}
\title{Prelim Outline}
\author{Shivesh Pathak}
\usepackage[margin=1.0in]{geometry}
\usepackage{graphicx}
\usepackage{float}
\usepackage{xcolor}
\begin{document}

\section{Introduction}
\begin{itemize}
\item Systematically developing model Hamiltonians which can accurately describe the energies and properties of the lowest lying eigenstates for strongly correlated electron systems is a pressing problem in modern condensed matter physics. 

\item The necessity for non-perturbative, many-body models in describing the low-lying excited states of strongly correlated electron systems poses a difficult challenge when developing effective theories for real world systems.

\item Common approaches to building suitable model Hamiltonians for strongly correlated materials involve using single particle theories like density functional theory (DFT).

\item We propose using a DMD procedure to develop accurate many-body effective theories for single (SLG) and bi-layer graphene (BLG) systems as a step towards building a many-body model for twisted bilayer graphene (TBLG).
\end{itemize}

\section{Methods}
\begin{enumerate}
\item Density matrix downfolding
\begin{itemize}
\item We begin by defining what we mean by a low-energy effective Hamiltonian and downfolding.

\item The key insight of DMD is that the Hamiltonian downfolding problem can be mapped onto an equivalent linear regression problem.

\item This linear regression problem can be tackled in three steps, beginning with sampling states from $\mathcal{LE}$.

\item Generally, one cannot sample the true low-energy subspace and an approximate subspace $\mathcal{LE}^\prime$ is sampled.

\item Next, a set of candidate descriptors which form the effective Hamiltonian are selected.

\item Typically, candidate operators are written in second quantized form, and an appropriate single particle basis to express these operators must be constructed.

\item Finally, the sampled data are used to fit the coefficients $\{g_k\}$ by linear regression.

\item Importantly, the fit effective Hamiltonian carries with it a quantitative measure of its validity.
\end{itemize}

\item Fixed-node diffusion Monte Carlo 
\begin{itemize}
\item Diffusion Monte Carlo (DMC) is a quantum Monte Carlo method which projects out the ground state of a real-space Hamiltonian given some initial trial wave function.

\item This stochastic implementation of DMC suffers from a fermion sign problem which is alleviated via a fixed-node approximation.

\item We take advantage of the variational nature of FN-DMC to sample the low-energy states necessary for DMD.
\end{itemize}
\end{enumerate}

\section{Preliminary results and discussion}
\paragraph{C2 Non-orthogonal: Brief summary} In order to become acquainted with our code and QMC algorithms in general I worked on implementing non-orthogonal orbital optimization for QMC trial wave functions and testing the effectiveness of these trial functions in calculating the total ground state energy of a C$_2$ molecule.

\paragraph{CuO: Introduction} As a first step towards developing accurate models for extended systems like solids using DMD, we constructed a many-body effective model for the CuO molecule which accurately describes the eigenstates and spectra seen in experiment up to 2eV above the ground state. 

\paragraph{CuO: Methods} In order to fit a model using DMD we had to construct a low energy space (LES), choose a sampling scheme for states within the LES resulting in our sample space (SS), develop a basis which can accurately describe variations within the LES, identify and select descriptors built out of these basis elements, and appropriately fit the data from our SS onto our selected descriptors. 

\paragraph{CuO: Results and Discussion} The resultant model from our analysis accurately describes the various eigenstates as measured in experiment up to 2eV, and is quantitatively and qualitatively more accurate than competing methods like density functional theory.

\paragraph{Hubbard: Preliminary results} Our preliminary calculations on the 2x2 half-filled 1-band Hubbard model map out the limiting behavior in the U-T model space by investigating the effectiveness of the single descriptor models. \textcolor{red}{Will be changed based on how far along we get with preliminary results.}

\section{Proposed work}
\begin{itemize}
\item My proposed plan involves building a sequence of model Hamiltonians for increasing complicated systems using DMD, working towards an accurate many-body model for TBLG.

\item The first system I will develop a model for is the benzene molecule.

\item While relatively simple, this molecule shares many similarities with and forms the basic unit of SLG.

\item As such, developing a model for benzene will help me become familiar with model fitting for 2-D carbon based systems.

\item The candidate descriptors I will use for DMD are motivated by a previous calculation on the benzene molecule.

\item A new constrained variational Monte Carlo (CVMC) method will be used to sample the low-energy wave functions necessary for DMD.

\item Note also that a model for benzene has already been developed using DMD.

\item This makes benzene a useful system not only for my personal learning, but as a benchmarking system for new methods like CVMC which have been developed recently.

\item The second system I will be interested in working on is SLG.

\item The model I am interested in working on will be an extension of a previous model developed for SLG using DMD.

\item I will extend the previous model by introducing long range density-density interactions into the effective theory.

\item The candidate model I anticipate to work with is an extended Hubbard model of the form.

\item CVMC will be used to sample the low-energy wave functions necessary to fit the model.

\item The last systems I will develop models for are AA and AB stacked BLG.

\item Being the simplest BLG configuration, these models will form a starting point for a more thorough development of models for TBLG.

\item I anticipate that the candidate descriptors should include interlayer couplings as well as the terms seen in SLG.

\item I will again use CVMC to generate the low-energy wave functions for the model regression.

\item The final four models can be used to investigate the transferrability of model parameters between these different carbon-based systems.

\item I will conclude by briefly discussing potential avenues of further study in using DMD to develop model Hamiltonians for TBLG. 

\end{itemize}
\section{Proposed timeline}
\textit{Do this after proposed work nailed down}.

\end{document}


