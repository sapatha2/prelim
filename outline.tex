\documentclass{article}
\title{Prelim Outline}
\author{Shivesh Pathak}
\usepackage[margin=1.0in]{geometry}
\usepackage{graphicx}
\usepackage{float}
\usepackage{xcolor}
\begin{document}

\section{Introduction}
\begin{itemize}
\item Developing model Hamiltonians for realistic materials is a fundamental problem in modern condensed matter physics.

\item Yet unresolved is a paradigm for building many-body model Hamiltonians which can describe realistic systems.

\item Common approaches to building many-body theories for real materials involve density functional theory (DFT).

\item I propose using a quantum Monte Carlo (QMC) density matrix downfolding (DMD) procedure to develop accurate many-body model Hamiltonians for single (SLG) and bi-layer (BLG) graphene systems.
\end{itemize}

\section{Methods}
\begin{enumerate}
\item Density matrix downfolding
\begin{itemize}
\item I begin by defining what a low-energy effective Hamiltonian and downfolding are.

\item The key insight of DMD is that the Hamiltonian downfolding problem can be mapped onto a linear regression problem.

\item This linear regression problem can be tackled in three steps, beginning with sampling states from $\mathcal{LE}$.

\item Next, a set of candidate descriptors which form the effective Hamiltonian are selected.

\item Finally, the sampled data are used to fit the coefficients $\{g_k\}$ by linear regression.

\item Importantly, the fit effective Hamiltonian carries with it a quantitative measure of its validity.
\end{itemize}

\item Fixed-node diffusion Monte Carlo 
\begin{itemize}
\item Diffusion Monte Carlo (DMC) is a quantum Monte Carlo method which projects out the ground state of a real-space Hamiltonian given some initial trial wave function.

\item This stochastic implementation of DMC suffers from a fermion sign problem which is alleviated via a fixed-node approximation.

\item One can take advantage of the variational nature of FN-DMC to sample the low-energy states necessary for DMD.
\end{itemize}
\end{enumerate}

\section{Preliminary results and discussion}
\paragraph{C2 Non-orthogonal: Brief summary} In order to become acquainted with our code and QMC algorithms in general I worked on implementing non-orthogonal orbital optimization for QMC trial wave functions and testing the effectiveness of these trial functions in calculating the total ground state energy of a C$_2$ molecule.

\begin{enumerate}
\item Intro (no subsection)
\begin{itemize}
\item As a first step towards developing accurate models for extended systems like solids using DMD, I constructed a many-body effective model for the CuO molecule by downfolding (1) under a Born-Oppenheimer approximation with fixed bond length $r_e$ = 1.725\r{A}.
\end{itemize}

\item Sampling low-energy states
\begin{itemize}

\item For the CuO molecule, low-energy states were generated by
using FN-DMC projected multi-Slater-Jastrow (MSJ) trial wave functions.

\item The determinants were constructed from the orbitals of independent symmetry-targeted unrestricted Kohn Sham (UKS) calculations.

\item A shell sampling scheme was used to sample states from the selected approximate low-energy space.

\item The final sampled MSJ states were projected using FN-DMC to generate our low-energy states.
\end{itemize}

\item Selecting candidate descriptors 
\begin{itemize}
\item We selected the following set of 1- and 2-body operators to construct our candidate descriptors based on our understanding of the low-energy excitations of the CuO molecule and our sampled states

\item The parantheses of \textbf{eqref} contain the full set of symmetry allowed 1-body operators within the Cu 3d, 4s and O 2p space
\end{itemize}

\item Fitting the effective Hamiltonian 
\begin{itemize}
\item {A direct attempt at fitting all eleven terms in \textbf{eqref} leads to a high goodness of fit but unexpectedly large coefficients for the fit model.}

\item A principal component regression (PCR) was used to alleviate this multicollinearity and fit the effective theory.

\item From the PCA, the nine highest ranked PCs were selected to use in the PCR.

\item {The model fit to these nine PCs maintains an excellent goodness of fit without serious multicollinearity.}
\end{itemize}

\item Solutions of effective Hamiltonian
\begin{itemize}
\item Solutions of our effective Hamiltonian are obtained by exact diagonalization.

\item The model is solved by exact diagonalization, and the solutions
agree well with experimental spectra and state assignments.

\item The model states also agree with recent theoretical calculations like UKS and CCSD(T).
\end{itemize}
\end{enumerate}

\section{Proposed work}
\begin{itemize}
\item My proposed plan involves building a sequence of model Hamiltonians for increasingly complicated graphene-like systems using QMC DMD, ending with BLG.

\item The first system I will develop a model for is the benzene molecule, shown in Figure 2a.

\item The candidate descriptors I will use for DMD are motivated by a previous calculation on the benzene molecule.

\item The second system I will be interested in working on is SLG.

\item In particular, I will work to extend a previous on-site Hubbard model developed for SLG by including long range density-density interactions.

\item The last systems I will develop models for are AA and AB stacked BLG, shown in Figure 2b.

\item I anticipate that the candidate descriptors should include interlayer couplings as well as the terms seen in SLG.

\item A new constrained variational Monte Carlo (CVMC) method will be used to sample the low-energy wave functions necessary for DMD.

\item I will use a multi-Slater-Jastrow (MSJ) parameterization in the CVMC sampling for each system.

\item Apart from their individual value, the four models can be used to investigate the transferrability of model parameters between these different carbon-based systems.

\item Further, my calculations will help mature methods like CVMC for further usage on strongly correlated systems.

\item Lastly, my calculations will lay the groundwork for model development for the strongly correlated TBLG system using DMD.
\end{itemize}
\section{Proposed timeline}


\end{document}


