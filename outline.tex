\documentclass{article}
\title{Prelim Outline}
\author{Shivesh Pathak}
\usepackage[margin=1.0in]{geometry}
\usepackage{graphicx}
\usepackage{float}
\usepackage{xcolor}
\begin{document}

\section{Introduction}
\begin{itemize}
\item Systematically developing model Hamiltonians which can accurately describe the energies and properties of the lowest lying eigenstates for strongly correlated electron systems is a pressing problem in modern condensed matter physics. 

\item The necessity for non-perturbative, many-body models in describing the low-lying excited states of strongly correlated electron systems poses a difficult challenge when developing effective theories for real world systems.

\item Common approaches to building suitable model Hamiltonians for strongly correlated materials involve using single particle theories like density functional theory (DFT).

\item We propose using a DMD procedure to develop accurate many-body effective theories for single (SLG) and bi-layer graphene (BLG) systems as a step towards building a many-body model for twisted bilayer graphene (TBLG).
\end{itemize}

\section{Methods}
\begin{enumerate}
\item Density matrix downfolding
\begin{itemize}
\item We begin by defining what we mean by a low-energy effective Hamiltonian and downfolding.

\item The key insight of DMD is that the Hamiltonian downfolding problem can be mapped onto an equivalent linear regression problem.

\item This linear regression problem can be tackled in three steps, beginning with sampling states from $\mathcal{LE}$.

\item Generally, one cannot sample the true low-energy subspace and an approximate subspace $\mathcal{LE}^\prime$ is sampled.

\item Next, a set of candidate descriptors which form the effective Hamiltonian are selected.

\item Typically, candidate operators are written in second quantized form, and an appropriate single particle basis to express these operators must be constructed.

\item Finally, the sampled data are used to fit the coefficients $\{g_k\}$ by linear regression.

\item Importantly, the fit effective Hamiltonian carries with it a quantitative measure of its validity.
\end{itemize}

\item Fixed-node diffusion Monte Carlo 
\begin{itemize}
\item Diffusion Monte Carlo (DMC) is a quantum Monte Carlo method which projects out the ground state of a real-space Hamiltonian given some initial trial wave function.

\item This stochastic implementation of DMC suffers from a fermion sign problem which is alleviated via a fixed-node approximation.

\item We take advantage of the variational nature of FN-DMC to sample the low-energy states necessary for DMD.
\end{itemize}
\end{enumerate}

\section{Preliminary results and discussion}
\paragraph{C2 Non-orthogonal: Brief summary} In order to become acquainted with our code and QMC algorithms in general I worked on implementing non-orthogonal orbital optimization for QMC trial wave functions and testing the effectiveness of these trial functions in calculating the total ground state energy of a C$_2$ molecule.

\begin{enumerate}
\item Intro (no subsection)
\begin{itemize}
\item Our objective is to construct a model Hamiltonian for the low-energy electronic excitations of the CuO molecule.

\item We will construct such a low-energy Hamiltonian by downfolding the non-relativistic \textit{ab-initio} Hamiltonian subject to the Born-Oppenheimer approximation.

\item It is important to note that the approximations present in (eqref) will prevent our model Hamiltonian from describing effects like bond-length relaxation and spin-orbit splitting.

\item The following four sections are dedicated towards detailing the DMD procedure.
\end{itemize}

\item Sampling low-energy states
\begin{itemize}
\item Sampling low-energy states involves selecting an approximate low-energy space to work with and choosing a scheme for drawing wavefunctions from that space.

\item We suggest two criteria for selecting $\mathcal{LE}^\prime$ so that a model fit on it approximates the target model fit on $\mathcal{LE}$ accurately.

\item First, the approximate space should be able to capture most of the variation among states expected in $\mathcal{LE}$.

\item Second, $\mathcal{LE}^\prime$ should be as low in energy as possible.

\item The sampling scheme can introduce a sampling bias into the final regressed model, and must be carefully chosen.

\item In this work, we consider an approximate low-energy space of FN-DMC projected multi-Slater-Jastrow trial wave functions.

\item The first criteria above can be satisfied by selecting determinants which capture the low-energy variations seen in the CuO molecule.

\item We can capture these variations by using determinants resulting from independent symmetry-targeted unrestricted Kohn Sham (UKS) calculations.

\item Of these UKS determinants, only those which contribute to accurately describing the low-lying excitations of the molecule are selected.

\item The second criteria is met by the FN-DMC projection.

\item A shell sampling scheme was used to sample states from the selected approximate low-energy space.

\item Extra samples were drawn for regions of descriptor space with sparse sample density to ensure a uniform distribution of samples across the space.
\end{itemize}

\item Selecting candidate descriptors 
\begin{itemize}
\item We start by constructing a single particle basis which will be used to express second quantized candidate operators based on two criteria.
First, the basis elements should be localized, atomic-like, and mutually orthogonal.

\item Second, the trace of the FN-DMC 1-RDM evaluated on the single particle basis should approximately equal the total number of electrons in the system.

\item To satisfy the first criteria above, we construct a set of intrinsic atomic orbitals (IAOs) to use for our single particle basis.

\item We validate the quality of the IAO basis by evaluating the trace of the FN-DMC 1-RDM on the basis over our sampled data.

\item We select the following set of 1- and 2-body operators to construct our candidate descriptors based on our understanding of the low-energy excitations of the CuO molecule and our sampled states

\item The parantheses of \textbf{eqref} contain the full set of symmetry allowed 1-body operators within the Cu 3d, 4s and O 2p space

\item The operator $\hat{n}_{3d_\delta}$ is not included as it is linearly dependent on the rest of the occupation operators

\item Two 2-body interaction terms are also considered: a Hund's coupling between Cu 4s and 3d orbitals and a Cu 4s Coulomb repulsion
\end{itemize}

\item Fitting the effective Hamiltonian 
\begin{itemize}
\item {A direct attempt at fitting all eleven terms in \textbf{eqref} leads to a high goodness of fit but unexpectedly large coefficients for the fit model.}

\item {By studying the condition number (CN) associated with the fit we conclude the large coefficients are symptomatic of multicollinearity between our candidate descriptors.}

\item {Alleviation of this multicollinearity is necessary as we will be interested in inferences related to the coefficients of our final model.}

\item {We approach the fitting using a principal components regression (PCR) to reduce the impact of the multicollinearity among our candidate descriptors.}

\item {Before moving forwards with the PCR we discuss two quantities used in our principal component selection: the cumulative explained variance ratio (CEVR) and CN.}

\item {We conduct a PCA and select the nine highest ranked PCs to use in the PCR.}

\item {A model fit to these nine PCs maintains an excellent goodness of fit without serious multicollinearity.}

\item {An inverse PCA transformation is used to map the fit coefficients corresponding to the PC descriptors back into the original eleven coefficients of (eqref), resulting in our final model.}

\item {We conclude by validating our model using a k-folds cross validation scheme.}

\item {We conduct 100 independent 5-fold cross validation tests for our nine PC model and find no indication of overfitting.}
\end{itemize}

\item Solutions of effective Hamiltonian
\begin{itemize}
\item Solutions of our effective Hamiltonian are obtained by exact diagonalization.

\item Before making comparisons to experiment, it was necessary to assign molecular term symbols to the model eigenstates.

\item We constructed the molecular orbitals (MOs) used for state assignments by diagonalizing the 1-body part of the model Hamiltonian.

\item We briefly discuss the constructed MOs, beginning with the $\delta$ and $\pi$ symmetry orbitals.

\item The three $\sigma$ symmetry MOs do not follow a simple bonding, non-bonding, anti-bonding structure.

\item In Figure \ref{fig:ed} we present the energies and occupations on this MO basis of the lowest-lying model eigenstates.

\item The model eigenstates are assigned term symbols using these MO occupations and spin multiplicities.

\item Equipped with term symbols, we compare our model solutions to experiment and find excellent agreement in state ordering and energies.

\item Note that our model has a larger error in describing high-energy eigenstates of the CuO molecule than low-energy ones.

\item The model states also agree with recent theoretical calculations like UKS and CCSD(T).

\item We believe the overestimation of the $^2\Delta$ excitation energy in UKS arises from an inadequate relaxation of the $\sigma$ orbitals.

\item We end by discussing the existence of a low-energy $^2\Sigma^-$ state for the CuO molecule which has not been observed in photoelectron spectroscopy.
\end{itemize}
\end{enumerate}

\section{Proposed work}
\begin{itemize}
\item My proposed plan involves building a sequence of model Hamiltonians for increasing complicated systems using DMD, working towards an accurate many-body model for TBLG.

\item The first system I will develop a model for is the benzene molecule.

\item While relatively simple, this molecule shares many similarities with and forms the basic unit of SLG.

\item As such, developing a model for benzene will help me become familiar with model fitting for 2-D carbon based systems.

\item The candidate descriptors I will use for DMD are motivated by a previous calculation on the benzene molecule.

\item A new constrained variational Monte Carlo (CVMC) method will be used to sample the low-energy wave functions necessary for DMD.

\item Note also that a model for benzene has already been developed using DMD.

\item This makes benzene a useful system not only for my personal learning, but as a benchmarking system for new methods like CVMC which have been developed recently.

\item The second system I will be interested in working on is SLG.

\item The model I am interested in working on will be an extension of a previous model developed for SLG using DMD.

\item I will extend the previous model by introducing long range density-density interactions into the effective theory.

\item The candidate model I anticipate to work with is an extended Hubbard model of the form.

\item CVMC will be used to sample the low-energy wave functions necessary to fit the model.

\item The last systems I will develop models for are AA and AB stacked BLG.

\item Being the simplest BLG configuration, these models will form a starting point for a more thorough development of models for TBLG.

\item I anticipate that the candidate descriptors should include interlayer couplings as well as the terms seen in SLG.

\item I will again use CVMC to generate the low-energy wave functions for the model regression.

\item The final four models can be used to investigate the transferrability of model parameters between these different carbon-based systems.

\item I will conclude by briefly discussing potential avenues of further study in using DMD to develop model Hamiltonians for TBLG. 

\end{itemize}
\section{Proposed timeline}
\textit{Do this after proposed work nailed down}.

\end{document}


