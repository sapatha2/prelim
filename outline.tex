\documentclass{article}
\title{Prelim Outline}
\author{Shivesh Pathak}
\usepackage[margin=1.0in]{geometry}
\usepackage{graphicx}
\usepackage{float}
\usepackage{xcolor}
\begin{document}

\section{Introduction and methods}
\paragraph{Overall goal} We want to be able to systematically develop low-energy effective models for strongly correlated electron systems.

\paragraph{Barrier to achieving goal} Non-perturbative interactions between various degrees of freedom - for example, lattice, spin and charge - makes writing down low-energy effective theories for strongly correlated materials very challenging.

\paragraph{Current state of the art} \textcolor{red}{Talk about methods people use for model fitting now. Will fill out as I do literature review.}

\paragraph{Advancement to the state of the art} We propose using a density matrix downfolding (DMD) procedure to develop low-energy effective theories of the 1-band nearest-neighbor Hubbard model on a square lattice as a stepping stone towards building effective theories for strongly correlated materials.

\paragraph{Density matrix down folding} 
The DMD procedure allows us to develop low-energy effective theories for physical systems in a systematic manner beginning with a higher energy Hamiltonian, like the \textit{ab-initio} Hamiltonian $H_{ab}$.

\paragraph{Fixed node diffusion Monte Carlo}
We will use fixed node diffusion Monte Carlo (FN-DMC) to generate wave functions in our chosen low-energy subspace and to accurately calculate \textit{ab-initio} energies and reduced density matrices.

\section{Preliminary results and discussion}
\paragraph{C2 Non-orthogonal: Brief summary} In order to become acquainted with our code and QMC algorithms in general I worked on implementing non-orthogonal orbital optimization for QMC trial wave functions and testing the effectiveness of these trial functions in calculating the total ground state energy of a C$_2$ molecule.

\paragraph{CuO: Introduction} As a first step towards developing accurate models for extended systems like solids using DMD, we constructed a many-body effective model for the CuO molecule which accurately describes the eigenstates and spectra seen in experiment up to 2eV above the ground state. 

\paragraph{CuO: Methods} In order to fit a model using DMD we had to construct a low energy space (LES), choose a sampling scheme for states within the LES resulting in our sample space (SS), develop a basis which can accurately describe variations within the LES, identify and select descriptors built out of these basis elements, and appropriately fit the data from our SS onto our selected descriptors. 

\paragraph{CuO: Results and Discussion} The resultant model from our analysis accurately describes the various eigenstates as measured in experiment up to 2eV, and is quantitatively and qualitatively more accurate than competing methods like density functional theory.

\paragraph{Hubbard: Preliminary results} Our preliminary calculations on the 2x2 half-filled 1-band Hubbard model map out the limiting behavior in the U-T model space by investigating the effectiveness of the single descriptor models. \textcolor{red}{Will be changed based on how far along we get with preliminary results.}

\section{Proposed work}
\paragraph{Hubbard: Effective models at half-filling} We want to develop low-energy effective theories for the half-filled Hubbard model over the space U: 0 - 12, 1/$k_B$T: 0 - 3 using the DMD method, particularly on the medium coupling region of U: 2 - 6 and 1/$k_B$T $\sim$ 1 where perturbative effective theories break down and a metal-insulator transition exists.

\paragraph{Hubbard: Effective models away from half-filling} A farther reaching goal would be to develop effective theories in the U-T-$\delta$ space using DMD, where more complex behavior similar to the cuprates have been found.

\paragraph{Hubbard: Finite size effects} Our initial calculations necessary for DMD can be done using exact diagonalization, but memory limitations will require us to use approximate methods like lattice quantum Monte Carlo to calculate energies and density matrix elements for states in our LES.

\section{Proposed timeline}
\textit{Do this after proposed work nailed down}.

\end{document}


