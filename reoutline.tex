\documentclass{article}
\begin{document}
\section{Introduction}
\begin{enumerate}
\item Developing model Hamiltonians for realistic materials is a fundamental problem in modern condensed matter physics.

\item Yet unresolved is a paradigm for building many-body model Hamiltonians which can describe strongly correlated systems.

\item In an attempt to build accurate many-body Hamiltonians for realistic systems, a framework for model development using solutions from \textit{ab-initio} simulations has emerged.

\item A recently developed density matrix downfolding (DMD) method provides a more direct link between \textit{ab-initio} and many-body effective Hamiltonians.

\item The utility of DMD is made apparent when studying the recent surge in model development for twisted bilayer graphene (TBLG).

\item As such, one portion of a project between research groups at UIUC, UCSB and Rice University is dedicated towards the development of an accurate interacting model for TBLG.
\end{enumerate}

\section{Density Matrix Downfolding (DMD)}
\begin{enumerate}
\item I will begin by defining carefully the terms model Hamiltonian and Hamiltonian downfolding.

\item The key insight of DMD is that the Hamiltonian downfolding problem is equivalent to a linear regression problem.

\item Being cast to a linear regression, the downfolding problem can be handled highly systematically.

\item Practically, the linear regression is tackled in three steps.

\item The through line to \textit{ab-initio} is maintained via the training data.
%QMC because accurate + efficient, but details later
\end{enumerate}

\section{Fixed-node diffusion Monte Carlo (FN-DMC)}
\begin{enumerate}
\item Diffusion Monte Carlo (DMC) is a quantum Monte Carlo method which projects out the ground state of a real-space Hamiltonian given some initial trial wave function.
%Talk about efficiency here 

\item DMC, however, suffers from a fermion sign problem which is alleviated via a fixed-node approximation.
%Talk about accuracy here, compare to other methods

\item While seemingly disadvantageous, the variational nature of FN-DMC plays a key role in generating the low-energy states needed for DMD.  

\item The training data required for downfolding are extracted from the projected low-energy states via a mixed estimator.
%Start paragraph with something else, probably the stochastic representation of the wave function
\end{enumerate}

\end{document}