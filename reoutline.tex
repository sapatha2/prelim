\documentclass{article}
\begin{document}
\section{Introduction}
\begin{enumerate}
\item Developing model Hamiltonians for realistic materials is a fundamental problem in modern condensed matter physics.

\item Yet unresolved is a paradigm for building many-body model Hamiltonians which can describe strongly correlated systems.

\item In an attempt to build accurate many-body Hamiltonians for realistic systems, a framework for model development using solutions from \textit{ab-initio} simulations has emerged.

\item A recently developed density matrix downfolding (DMD) method provides a more direct link between \textit{ab-initio} and many-body effective Hamiltonians.

\item The utility of DMD is made apparent when studying the recent surge in model development for twisted bilayer graphene (TBLG).

\item As such, one portion of a project between research groups at UIUC, UCSB and Rice University is dedicated towards the development of an accurate interacting model for TBLG.
\end{enumerate}

\section{Density Matrix Downfolding (DMD)}
\begin{enumerate}
\item I will begin by defining carefully the terms model Hamiltonian and Hamiltonian downfolding.

\item The key insight of DMD is that the Hamiltonian downfolding problem is equivalent to a linear regression problem.

\item Being cast to a linear regression, the downfolding problem can be handled highly systematically.

\item Practically, the linear regression is tackled in three steps.

\item The through line to \textit{ab-initio} is maintained via the training data.
%QMC because accurate + efficient, but details later
\end{enumerate}

\section{Fixed-node diffusion Monte Carlo (FN-DMC)}
\begin{enumerate}
\item Diffusion Monte Carlo (DMC) is a quantum Monte Carlo method which projects out the ground state of a real-space Hamiltonian given some initial trial wave function.
%Talk about efficiency here 

\item DMC, however, suffers from a fermion sign problem which is alleviated via a fixed-node approximation.
%Talk about accuracy here, compare to other methods

\item While seemingly disadvantageous, the variational nature of FN-DMC plays a key role in generating the low-energy states needed for DMD.  

\item The training data required for downfolding are extracted from the projected low-energy states via a mixed estimator.
%Start paragraph with something else, probably the stochastic representation of the wave function
\end{enumerate}

\section{Non-orthogonal determinants in FN-DMC trial wave functions}
\begin{enumerate}
\item In order to become acquainted with our code QWalk and QMC algorithms in general I worked on implementing and testing multi-Slater-Jastrow trial functions with optimized nonorthogonal determinants (MSJ+NO) in FN-DMC.

\item We assessed the efficiency and compactness of this new trial function by calculating the ground state energy and single particle densities of a C$_2$ molecule using FN-DMC and comparing to the results when using multi-Slater-Jastrow trial functions with optimized orthogonal determinant trial functions (MSJ+O).

\item Our results indicated that using non-orthogonal determinants may
lead to more compact multi-Slater-Jastrow trial wave functions for small molecules.
\end{enumerate}

\section{Effective theory for CuO molecule using DMD}
\begin{enumerate}
\item To become acquainted with DMD, I constructed a many-body effective model for the CuO molecule with fixed bond length r$_e$ = 1.725A.

%sampling low energy states
\item Explain why MSJ was chosen (Probably the hardest one)

\item How determinants selected

\item How coefficients selected 

\item How Jastrow optimized and FN-DMC done

%candidate descriptors 
\item The following set of 1- and 2-body operators were selected as candidates for the model based on understanding of the low-energy excitations of the CuO molecule and properties of the sampled low-energy states.

\item The first chunk of descriptors were selected using physical intuition and symmetry principles.

\item The inclusion of a double occupancy term $n_{4s,\uparrow} n_{4s,\downarrow}$ was motivated directly from the sampled low-energy states.

%Fitting the model 
\item A direct attempt at fitting all eleven terms in \textbf{eqref} leads to a high goodness of fit, R$^2$ = 0.99, but unexpectedly large coefficients for the fit model.

\item A principal components analysis (PCA) on the descriptors illustrates the multicollinearity in the data set.

\item To reduce multicollinearity while maintaining a high goodness of fit, I fit a model to these nine PCs.

%Solutions of the model 
\item The model is solved by exact diagonalization, and the solutions agree well with experimental spectra and state assignments.

\item The model states also agree with recent theoretical calculations like UKS and CCSD(T).

\end{enumerate}

\end{document}