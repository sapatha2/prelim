\documentclass{article}
\begin{document}
\section{Introduction}
\begin{enumerate}
\item Developing accurate model Hamiltonians for strongly correlated materials remains an outstanding problem in modern condensed matter physics.

\item To bridge this gap, a framework for model development using solutions from \textit{ab-initio} simulations has emerged.

\item A recently developed density matrix downfolding (DMD) method addresses the need for a more systematic link between \textit{ab-initio} and model Hamiltonians.
%No QMC here

\item The surge in model development for twisted bilayer graphene (TBLG) exemplifies the necessity for such a systematic framework.

\item I propose using DMD to systematically develop accurate, many-body model Hamiltonians for single- (SLG) and simple bi-layer (BLG) graphene systems.
%Start this paragraph with the grant proposal, have my thesis statement in the middle makes more sense!
%Name drop QMC
\end{enumerate}

\section{Density Matrix Downfolding (DMD)}
\begin{enumerate}
\item I will begin by defining carefully the terms model Hamiltonian and Hamiltonian downfolding.

\item The key insight of DMD is that the Hamiltonian downfolding problem is equivalent to a linear regression problem.

\item Being cast to a linear regression, the downfolding problem can be handled highly systematically.

\item Practically, the linear regression is tackled in three steps.

\item The through line to \textit{ab-initio} is maintained via the training data.
%QMC because accurate + efficient, but details later
\end{enumerate}

\section{Fixed-node diffusion Monte Carlo (FN-DMC)}
\begin{enumerate}
\item Diffusion Monte Carlo (DMC) is a quantum Monte Carlo method which projects out the ground state of a real-space Hamiltonian given some initial trial wave function.
%Talk about efficiency here 

\item DMC, however, suffers from a fermion sign problem which is alleviated via a fixed-node approximation.
%Talk about accuracy here, compare to other methods

\item While seemingly disadvantageous, the variational nature of FN-DMC plays a key role in generating the low-energy states needed for DMD.  

\item The training data required for downfolding are extracted from the projected low-energy states via a mixed estimator.
%Start paragraph with something else, probably the stochastic representation of the wave function
\end{enumerate}

\end{document}