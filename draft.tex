\documentclass{article}
\usepackage{graphicx}
\usepackage{xcolor}
\usepackage{subcaption}
\usepackage[margin=1.0in]{geometry}
\usepackage{float}
\usepackage{ulem}
\usepackage{amsmath}
\usepackage{mathtools}

\begin{document}
\section{Introduction}
A core problem of condensed matter physics is understanding how to systematically develop low-energy effective models for strongly correlated electron systems. In the forefront of modern research are high temperature superconductors, heavy fermion materials, materials with strong spin-orbit coupling effects, and more for which simple effective theories have not yet been found. For example, cuprate family of high-Tc superconductors have an intricate phase diagram in the doping-temperature space \textbf{ref} with a robust anti-ferromagnetic (AFM) charge-transfer insulator state near zero doping, a prominent superconducting dome in the 0.1 - 0.2 hole doped region at low temperatures with psuedo-gap and non-Fermi liquid metallic phases at higher temperatures and Fermi liquid like behavior in the overdoped area, of which only the AFM insulating and Fermi liquid phases have well understood mechanism and models \textbf{ref}. The iron-pnictides also have superconducting domes, non-Fermi liquid like metallic phases, a spin-density-wave antiferommagnetic phase, all under various dopings and temperatures \textbf{ref}. As in the cuprate case, models for these phases have not been fully fleshed out \textbf{ref}. Heavy-fermion materials have a purported quantum critical point between an AFM and Fermi-liquid phase which houses a superconducting dome, leading to another phase diagram for which building model Hamiltonians is very challenging \textbf{ref}. Materials with strong spin-orbit coupling effects exhibit a host of properties like \textbf{LIST THESE HERE!}. 

The fundamental barrier in writing down low-energy theories for strong correlated materials is the existence of non-perturbative interactions between various degrees of freedom - for example, lattice, spin and charge. Consider just the cuprate case for now, the two phases which are well understood contain a singular important degree of freedom in the low-energy space. Near zero doping the cuprates are charge-transfer insulators with an optical gap around 2 eV \textbf{ref}, indicating that the lowest energy excitations are primarily spin-like, allowing for a simple nearest-neighbor spin-only model like an AFM Heisenberg model to describe the low-energy physics quite accurately. In the heavily overdoped side, the lowest energy excitations look like well defined Fermi-liquid quasiparticles \textbf{ref}, indicating our lowest energy excitations are electron-like, and a simple non-interacting electron model (with renormalized masses, etc) will suffice. In the intermediate doping region where the low-energy excitations cannot be easily sectioned into spin-like or electron-like only, we see the emergence of exotic phases and where the most difficulties arise in writing down low-energy effective models. On top of that, this entire discussion did not include any lattice effects like structural phase transitions which many strongly-correlated materials have as well \textbf{ref}. This discussion can be extended to the many other strongly correlated electron materials presented above as well.

\textbf{Current state of the art.}

We propose using a density matrix downfolding (DMD) procedure\textbf{ref} to develop low-energy effective theories of the 1-band nearest-neighbor Hubbard model on a square lattice as a stepping stone towards building effective theories for strongly correlated materials. The Hubbard model is a simple model with nearest-neighbor hopping and a local on-site Coulomb repulsion, parameterized by a single ratio U/t: 
\begin{equation}
H_\text{hub} = -t \sum_{\langle i,j \rangle,\sigma}( c_{i^\dagger,\sigma} c_{j,\sigma} + h.c.) + U \sum_i n_{i\uparrow} n_{i\downarrow}
\label{hubbard}.
\end{equation}
The solutions in 1-d are well understood via the Bethe ansatz \textbf{ref}, however the 2-d square lattice Hubbard model has no known exact solution.  In the very strong coupling limit $U/t \rightarrow \infty$ an emergent low-energy spin-space exists with an effective nearest-neighbor AFM Heisenberg Hamiltonian \textbf{ref}. In the region $U = 0$ a trivial non-interacting theory holds. Between these two limits, in particular the intermediate coupling region $1 < U < 7$, no well established effective theory exists. In this region there is a metal-insulator transition at $U_c/t \sim 4$ \textbf{ref}, and the pseudogap opens up at finite temperature around $U/t \sim 7$, leading to the development of a Mott gap for large $U/t > 11$ \textbf{ref}. In addition to the half-filling phase diagram, this model exhibits a plethora of interesting ground state properties when doped including stripe phases and d-wave pairing \textbf{ref}. The relative simplicity of the Hubbard model and the lack of a well-established low-energy theory in a broad intermediate coupling region lends the model as a useful playground for our new downfolding procedure in preparation for tackling realistic strongly correlated electron systems. 

The DMD procedure allows us to develop low-energy effective theories
for physical systems in a systematic manner beginning with a higher energy Hamiltonian H on a Hilbert space $\mathcal{H}$. When trying to develop effective theories for realistic systems H is usually the \textit{ab-initio} Hamiltonian
\begin{equation}
H_\text{ab} = -\frac{1}{2} \sum_{i} \nabla_i^2 - \sum_{i,I}\frac{Z}{|r_i - R_I|} + \sum_{i<j}\frac{1}{|r_i - r_j|} 
\label{ab}
\end{equation} 
where the sum $i$ is over electrons and $I$ over nuclei, and the unit for energy is Hartree (Ha). Typically we work under the Born-Oppenheimer approximation and use pseudo-potentials as an efficient replacement for core electrons. This is not always the case, for example when downfolding a second quantized model like shown in equation \ref{hubbard}. The core idea behind the DMD procedure is functional fitting, in particular approximating the total energy functional on a particular subspace of the full Hilbert space. Mathematically:

\begin{equation}
(E[\Psi] \equiv \langle \Psi | H |\Psi \rangle, \mathcal{H}) \rightarrow (E_{eff}[\Psi], \mathcal{LE})\text{ s.t. } E_{eff}[\Psi] = E[\Psi] + \epsilon[\Psi] \ \forall \ |\Psi\rangle \in \mathcal{LE}.
\label{DMD}
\end{equation}
where $\epsilon$ represents an error within our effective theory. The fitting procedure is conducted similarly to fitting any function: 1) Choose a $\mathcal{LE}$ which is spanned by the N lowest energy eigenvectors within $\mathcal{H}$, 2) sample the space to get a sample set $\mathcal{SS} \in \mathcal{LE}$, 3) construct a set of potential models with some variable parameters $\mathcal{MS}$, and 4) regress these models using your sampled data to fit the variable parameters. The density-matrix part of DMD comes from the particular choice of functional parameterization: 
\begin{equation}
E_{eff}[\Psi] = \sum_i c_i d_i[\Psi],\ d_i[\Psi] = \langle \Psi |\hat{d}_i|\Psi \rangle
\end{equation}
where $\hat{d}_i$ is a Hermitian operator in our Hilbert space and many times chosen to be a (sum of) 1- or 2-body density matrices. Common choices of $\hat{d}_i$ are $n_k,\ c_j^\dagger c_k + h.c.,\ n_{k\uparrow}n_{k\downarrow},\ \vec{S}_k \cdot \vec{S}_l,\ $ etc. With this parameterization one can even extract an effective Hamiltonian from the effective energy functional: $H_{eff} = \sum_i c_i \hat{d}_i$ which can be solved and studied. Mapping a downfolding problem onto a regression problem allows us to develop methodologies and systems for building effective theories. In addition we can quantiatively validate our proposed effective theories using tools from data science and statistics. Not only is this method pleasing to statisticians, it is in principle exact. Suppose we want to build a model on $\mathcal{LE}$ composed of the N lowest-energy eigenstates. Let us consider a potential model $H_{eff} = \sum_{i=1}^{ N} c_i |\Psi_i\rangle \langle \Psi_i|$ where the sum is over the N lowest eigenstates with variable coefficients $\{c\}$. If we select \textit{any} linearly independent set of N samples from $\mathcal{LE}$, eigenstates are not required, an ordinary linear regression will lead to the optimal coefficients $c_i = E_i$, the eigenvalues of the eigenvectors $|\Psi_i\rangle$, and our effective model will be exact. In practice we do not know all the eigenstates of H so identifying and sampling $\mathcal{LE}$ is of primary concern, especially for strongly correlated electron systems where the low-energy degrees of freedom may be very complicated collective excitations. I will discuss one approach we had to sampling the low-energy space when talking about our model for the CuO molecule later on. If we do not know the eigenstates we also need to choose descriptors $d_i$ which can compactly represent the variation in properties within $\mathcal{LE}$. Typically one will collect all reasonable $d_i$ like occupation, hopping, exchange coupling, on-site Coulomb repulsion, Hund's coupling, etc ad consider only those $d_i$ which vary within $\mathcal{SS}$. One can then use tools from data science like principal component analysis, step-wise AIC/BIC, regularized regression (e.g. LASSO) for model selection and fitting. 

When downfolding \textit{ab-initio} Hamiltonians we will use fixed node diffusion Monte Carlo (FN-DMC) to generate wave functions in our chosen low-energy subspace and to accurately calculate energies and reduced density matrices. Diffusion Monte Carlo (DMC) is a quantum Monte Carlo method which projects out the ground state of a Hamiltonian given some initial trial wave function. Consider a trial wave function $|\psi\rangle$ and a Hamiltonian $H$ with ground state $|\phi_0\rangle$. We apply the projector $e^{-\tau (H-E_0)}$ as $\tau \rightarrow \infty$ to $|\psi \rangle$
\begin{equation}
\lim_{\tau \rightarrow \infty} e^{-\tau (H-E_0)} |\psi\rangle \propto \langle \phi_0|\psi\rangle |\phi_0\rangle,
\end{equation}
projecting out the ground state $|\phi_0\rangle$ of $H$ as long as the trial wave function we choose is not orthogonal to the ground state. The stochastic implementation involves moving samples from the trial function $\Psi_T(R)$ using the Green function $G(R, R^\prime, \tau) = \langle R | e^{-\tau(H - E_T)} | R^\prime \rangle$. Since the Hamiltonian has two parts, $H = T + V$ a Trotter expansion is used to approximate the Green function $G(R, R^\prime, \tau) = \langle R | e^{-\tau(H - E_T)} | R^\prime \rangle \sim e^{-\tau(V(R) + V(R^\prime))/2} \langle R| e^{-\tau(T - E_T)}|R^\prime \rangle + O(\tau^2) $ which leads to a time step error in the calculation if the time step between moves $\tau$ is too large. This implementation when used on a Hamiltonian with fermions also leads to a fermion sign problem. We deal with this sign problem through a fixed-node approximation, where the nodal surface of the projected wave function is forced to match that of the initial trial wave function. This approximation makes FN-DMC variational, and will only return the exact ground state of $H$ if the nodal surfaces of $|\psi\rangle$ and $|\phi_0\rangle$ are identical. Therefore the FN-DMC calculation has two major systematic errors that have to be dealt with, the fixed-node error and the time-step error. The time-step error is typically handled through a time-step extrapolation, where multiple FN-DMC calculations are conducted at sequentially smaller $\tau$ and the relevant expectation values are extrapolated to $\tau \rightarrow 0$ \cite{Needs2010}. Fixed-node error can be reduced by using optimized many-body trial wave functions of the (multi-)Slater-Jastrow form seen in \cite{PhysRevLett.98.110201}. We can use the final FN-DMC state to calculate the expectation values of operators like the Hamiltonian or reduced density matrices. For an operator $\hat{O}$ we will use a mixed estimator to calculate the expectation value $\langle \Psi_{DMC} |\hat{O} | \Psi_T \rangle/\langle \Psi_{DMC} | \Psi_T \rangle$ where $\Psi_{DMC}$ is the projected FN-DMC state starting with the trial wave function $\Psi_T$, as this estimator allows us to use the importance sampled distribution $f = \Psi_{DMC}\Psi_T$ for our Monte Carlo estimates. For operators which do not commute with the Hamiltonian H the mixed estimator has an error which can be corrected to first order by subtracting twice the VMC estimate of the expectation value \cite{ceperley_kalos_1979}. Wagner provides the details for calculating reduced density matrix elements in QMC \cite{doi:10.1063/1.4793531}.
\end{document}